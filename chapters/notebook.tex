
\chapter*{Notebook}

4 August 2017\\

Why am I doing this? Or if you're someone other than me reading this, why did I do this? Why write a book? And what is this part? I'll start with the second question.

Unlike all of those people who wrote books in the past, I'm writing this book now, so that makes me a modern writer, and everyone knows that modern writers have to pretend like they're reinventing the form. Usually this just means combining things other people did, so this part is my contribution to that tradition. At the end of the story, I'll put this notebook in the back of the book and it will contain notes I write as I'm writing this book.

So here's where I am. I just turned 30 on July 17 and started writing. I've been thinking about writing for a while, but I don't really do it. I like reading though and had finished all the books I had, so I started just putting the words down on my birthday since it seemed appropriate. A couple of weeks before that is when I started thinking seriously about writing a book and I was thinking it might be a science fiction novel in a hopeful near future with a character who invents the warp drive. But I know about CERN and physicists and I wanted to write a book about them instead.

So why write this book? There are a lot of misconceptions about physicists, and who we are and what we do. The misconception is that we're different. Really the misconception is that anyone's different, but we don't need to get into that right now. Of course we're all unique, but we're all also just people. So I wanted to talk about physicists.

I also wanted to talk about physics. Instead of science fiction, maybe this book is scientific fiction. The advice Adam has but doesn't follow when talking to a public audience, that you can never go too shallow, is something I have heard again and again. There´s truth to it, and it can be easy to lose your audience when you're talking to people who know nothing about a subject that you know intimately. But what's presented to us also sets a limit and people consistently demonstrate that they can understand complex topics when they're motivated.  

So I want to talk about physics, but actually talk about physics. I want to talk about particle physics more or less the way particle physicists talk about it, and I want to explain concepts that I think are important or interesting to nonscientists who think science is interesting. Scientists too. I'm about 10 pages into Alice's first chapter and had a really enjoyable lunch reading my old notes on the Young tableaux and reworking my solution to what was a real homework problem.

There are a lot of other things I want to talk about too. I'm writing this as an american citizen who has been living around CERN for a little over four years and it's 2017. You know what that means. Talking about people means talking about a lot of things, and my goal is to approach topics directly, through people's thoughts. I have a tendency to be a little controversial sometimes, so before I've written anything controversial, and maybe I won't, I would just like to point out that I don't necessarily agree with all of the thoughts my characters have. 

I want to talk about gender. Yes I'm male, but it turns out male is a gender too. I want to talk about race and culture and internationalism. I want to talk about bias and ambition, depression and friendship. And I want to talk about love. Love toward that one other person, or toward many people, or toward ideas or onesself. There's a lot to write about and I'm excited to see who these charactars end up being.
