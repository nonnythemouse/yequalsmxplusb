
\chapter*{Alice}

Alice and Bob are standing on flat ground on Earth, 20 meters apart. Alice can throw a ball at a speed of 5 meters per second. Can Alice throw a ball to Bob and if so, at what angle should Alice throw the ball?

Alice and Bob are pulling a 5 kilogram sled with ropes attached to the center front. Alice pulls with 3 Newtons of force on her rope which is at an angle of 30 degrees west of north and 15 degrees above the horizontal. Bob is pulling on his rope with a force of 5 Newtons at an angle of 15 degrees east of north and 30 degrees above the horizontal. How fast does the sled accelerate? Assume the coefficient of friction for snow is XXXXXXX. 

Alice and Bob are on trains, travelling away from each other at $3/5$ the speed of light. Alice can throw a ball at $4/5$ the speed of light. Assuming Alice and Bob are separated by 10 meters when Alice throws the ball, how far away will Alice observe Bob to be when she sees him catch the ball? How far away will Bob see Alice as being when he catches the ball?

Alice and Bob are twins. At the age of 20, Bob gets in a spaceship which travels at $3/5$ the speed of light away from Earth. After one year as measured on the spaceship, the ship turns around and comes back to Earth, again at $3/5$ the speed of light. Bob is now 22. How old is Alice? Does Bob recognize her?

Alice and Bob are scientists. Bob gets on a spaceship which sends out light pulses every second from his reference frame. As he crashes into a black hole, Alice sees the light pulses arrive further and further apart and misses her dear friend and colleague. Who designed this terrible experiment? Did Bob know his fate? Did he think himself a martyr for science? Could he have been tricked? Just told to man his station, ensuring those crucial light pulses made it out, every second on the second, while the rest of the crew abandonded ship? Was there a faulty valve? Some tiny overlooked component, like the O-rings that froze and became brittle which lead to the Challenger explosion? What was Bob doing out there? He wasn't the adventureous type. Couldn't they have just automated a computer to send the light pulses? Was Bob actually a cyborg? Had their ship been captured by the borg and now Bob didn't mind because the collective would get their data?

\mypause

Alice had awlays liked Alice and Bob problems, this ill-fated pair who loved playing catch in the most unusual circumstances. Sometimes they would play catch separated by a vertical wall of height $h$, or on different continents so that they had to take into account the curvature of the Earth. They would play catch between their spaceships, or on a frictionless, infinite pond.

Alice wondered if there were many different frictionless, infinite ponds, or did Alice and Bob just keep coming back to that same one? Did all of those imagined physics questions with people sliding around or balls coming to perfect elastic collisions, secretly take place on the same infinte pond? There was enough space, and if there was more than one infinite pond, they would all either have to be absolutely parallel, or they would intersect somewhere. What does it look like at the intersection of two infinite ponds? She started to draw a sketch of a small pond, surrounded by woods, with another pond angling in from the sky, only the surface visible, with ghostly trees from some other dimension whose branches intertwined with the trees around the pond where Alice and Bob sometimes played catch. 

\mypause

Alice had homework to do and she was procrastinating. They were coming up on the end of week seven and her problem sets for quantum and statistical mechanics would be due Monday and Tuesday. Midterms had been two weeks ago, where she got an A in quantum and a B+ in stat mech, and the end of the term was already nearly in sight. 

Everybody got As and Bs in grad school and she had once heard her friend Daniel refer to a B as a gentelman's F. Funny guy. And much smarter than he desreved to be. Not that he wasn't a good guy. He was nice and kind of goofy, but he never seemed to put any effort into math or physics and just understood the concepts immediately as they were explained. He was on track to be a theorist at some major university or exclusive research institute, maybe doing cosmology or string theory. Or maybe he'd just go get rich on Wall Street or Silicon Valley, at least for a little while, until he got bored and turned his genious somewhere else. He wasn't really the Wall Street type, and Alice could imagine him enjoying wearing a tie for about a week until he decided that it was more of a hassle than a game about tying knots. They'd make an exception for him, that he wouldn't have to wear one, but he'd leave anyways. It wasn't really about the tie.

\mypause

They had been in all their classes together so far and had met on the second day in the math building. Alice had decided to sit in on math courses that seemed interesting and was suprised to see Daniel in the front row of algebraic geometry when she walked in a minute before the hallway buzzer sounded. He hadn't noticed her, but as they were leaving, she went up to him and said that she recognized him from the physics orientation. 

"Oh yea. I thought you looked framiliar. Hi, I'm Daniel."

"I'm Alice," said Alice, "so what brings you to the math department?" It was suprisingly rare to see physics students in the math building, or math students in the physics building for that matter.

"I'm deciding between a few courses, but I'm only going to take one this term. It's between algebraic geometry and algebraic topology."

"Ha, what's the difference?" asked Alice, joking but not certain she could answer the question herself.

"I'm not totally sure actually," said Daniel. "I think geometry is going to be more group theory where topology is going to be more topology." He hadn't meant that last part ironically.

"I see," said Alice.

"Yep," said Daniel. "So what classes are you taking this term? Are you also in quantum?" Wow. Not a totally awkward physicist that only answered questions. He was trying to make conversation.

"Yea, I hear Professor Chang is real tough. He's a neutrino theorist that gives tests which even he can just barely finish in the alloted time."

"At least they're in class though," said Daniel. "An impossible take home exam can kill your whole weekend. If it's in class, at least you get it over with."

Alice hadn't thought about it like that. It's true though. If they wanted, their professors could give them an impossible test in class, take home, and even open notes if they wanted to. If they wanted to be mean. In class and with one handwritten note sheet, everyone was on the same playing field. You couldn't just try harder and stay up late, or find the solution online or in some old textbook in the library at the last minute. It was what you knew, what you could actually hold in your brain, that was being tested and that was also why it was so stressful.

Alice wasn't a good test taker. Mostly, she told herself, because she got nervious and couldn't concentrate with the clock counting down the seconds, everyone else immediately scribbling down their answers before she had even finished reading the first question. In reality, Alice usually did pretty well on tests, and finished her undergraduate with a 3.88 GPA, but knowing this still didn't help her when the door closed and the extra scrap paper was placed out at the front of the room.

Still, Alice always took a stack of at least twenty pages right from the beginning, just in case. Who knows? There's a story about some physics professor who would put an unsolved problem on all of his tests, until one day one of his worst students turned in his exam and eventually won the Fields medal for his answer to question three. Maybe it's not true but that kind of thing does happen sometimes. 

As they walked back to the physics department, they chatted about the other classes they were taking, it turns out they were both also in classical mechanics together, and about where they had done their undergrads, and where they were from, and that kind of thing, and from then on they would walk back to the physics building together after geometry most Mondays and Wednesdays at 11:25, chatting sometimes and sometimes just walking. Sometimes they would get lunch at the food carts which were set up in the lot behind Memorial Library. 

\mypause

Alice would be meeting Daniel on Sunday to talk about the stat mech problem set with some of the other students in the class. It had started just as Alice and Daniel, Daniel explaining partition functions, and them deciding that it would be easier if they took over an empty classroom to write on the board.

All of the first years were in the same hallway, and Sam and Joseph had poked their heads in on the way back from TAing and asked if they could join. Then Chris and then Alex and by their second week, the population had stabilized with about half the class showing up on Sunday afternoons to talk about the homework.

People would take turns up at the board, or sometimes two or three of them would work in parallel, each confident that theirs was the right approach, but also constantly looking over at what the others were writing down. Daniel usually lead things, and made vague statements that Alex would solidify into mathematics and write up on the board. Alex had very neat handwriting, and whenever he would pause, Alice or her classmates would start talking and telling him what to do next, or sometimes take their own piece of chalk and show him, or sometimes they would all just sit and wait, thinking, but also hoping someone else might have an idea. Hopefully Daniel.

In this way, the homework usually got done. Each problem was discussed and, since this was their life, they took as much time as they needed. Usually this meant all afternoon, Daniel would always leave before 5, but it meant until 3 in the morning at least one particularly brutal week, and well into the evenings about twice a month. It was kind of an honor system thing that everyone would have worked on the problems on their own before showing up on Sunday, but nobody ever checked or asked, and everyone copied nearly everything down so it was impossible to tell.

It was clear that some of them, like Alex, definitely had worked on the problems, and were usually just missing a few key pieces here or there. XXX GIVE REAL TRICKS: Differentiate your equations first and then try solving for the density, or BAH STATMECH XXXX Daniel claimed in the first few weeks that he hadn't worked on the problems beforehand, but made a killer study sheet for their midterm, and after that, usually showed up with at least some kind of notes. 

\mypause

Alice finished her sketch with seaweed trees that grew near the outlet of the real pond, which was still under the surface of the imaginary pond, and then a few more near that line of intersection, which passed through the rippled surface from one reality and into the other, only then starting to grow branches and buds. She had decided that it was spring there too, in the land of frictionless ponds, so the frictionless surfaces had melted and were quiet in the mornings.

She was a fast drawer, not deliberately trying to be quick, but she would lose herself and her sketches looked as though they had been exhaled onto the page in a single breath. For a while in college she had gotten interested in single line drawings, where you make the whole picture never separating pen from paper, but eventually decided the form was too restrictive. Paint required too much setup and materials, and pencil or charcoal would smudge, so she settled on black and white, pen and ink as her medium of choice and the margins of her textbooks were all the better for it. 

The same was true for physics homework, that pencil would smudge, and so it was also in college when she switched to using pen exclusively for that too. Everyone, it seemed to her, was always taught to use pencil so that you can erase your mistakes. But apart just from smudging, if you erased your mistakes you couldn't see them, so you're more likely to try the same wrong idea twice. Draw a line though what you need to and continue below. This meant that Alice usually ended up making two drafts of her homework, but by grad school, so was just about everyone else. 

\mypause

 {\ttfamily In the decomposition $2 \otimes 2 \otimes 2 = 2 \oplus 2 \oplus 4$, draw the Young tableaux for each of the 2, 2, and the 4. }

% $1$b) Recalling the rules for the addition of angular momentum, write the wave function for each of the states in the 2, 2, and the 4 in terms of the basis $2\otimes 2\otimes 2$.

\mypause 

She read the question again. 

\mypause
% Then she checked the math, $8=8$. Yep, works. Next problem? "Not so fast missy." She hadn't meant to say that out loud but sometimes things slipped out, especially when she was using her accent that was kind of half southern and half brittish. Maybe australian. They're like the brittish of the south. 
%
% And she wasn't working on stat mech, this was quantum. Group theory. The important stuff. The interesting stuff, once you got it. Math. She looked at 

Then she checked the math, $8=8$. Yep, works. But what was this even talking about? This wasn't stat mech, this was quantum. Group theory. The important stuff. The interesting stuff, once you got it. Math. 

She looked at the $\oplus$ symbols and only ever remembered seeing them in math classes, the kind of class which had to invent new symbols for their concepts, and got you to think about what numbers really were and not just how to use them. It was the symbol you used when you wanted to say two things were to be thought of as added, but since they weren't numbers you were adding, there was some other rule for what addition meant this time. So the symbol could really mean just about anything and the problem was saying that she needed to show that multiplying three pairs together was the same as adding two pairs and a quadruple, for whatever multiplication and addition meant in this context. Ok so she understood what she was being asked to do. Maybe? Either way, it didn't really help.

Wikipedia was good. Textbooks were better. This didn't stop anyone from going to Wikipedia first but as Alice progressed further and further into grad school, she had noticed that the articles she needed got shorter and shorter, or just wouldn't be there at all. Mostly they would try different keyword combinations on different search engines, looking for pdfs of lecture notes or homework solutions from other graduate physics departments. Lecture notes where they worked out your problem as an example were the best, but homework solutions were pretty good too, though it seemed like professors deliberately left out most of the steps, showing only the concept, stating that you should simply rotate your reference frame so the cross terms vanish, do your calculation, and then rotate back. Simply.

Finding one of those was a gem, but it was only once or twice that anyone came across the actual problem they were working on. And this wasn't really cheating, or at least nobody treated it like it was, professors included, though they never said it explicitly. It was almost research, developing the skills to find the resources you needed to answer the question at hand. 

No doubt the professors would have preferred it if they could all have just gone straight from their lecture notes to the problem sets, maybe with a few textbooks in there like they had used when they were students, but if you could understand how to solve the problem by the time the homework was due, then you had learned what they wanted you to learn and that was the whole point of homework in the first place. Plus grades really didn't matter unless you were failing out, and nobody was.

\mypause
 
The internet was kind to Alice. It seems the Young tableaux are the kind of thing professors love making 5 page worksheets about. It's a form of notation. It was invented by A. Young in 1901. It consists of drawing boxes to represent different particles that are going to be put together, with specific rules for how boxes or groups of boxes could be combined so that they obey the rules particles do, and then there are more rules for how to interpret the final pictures that come out. 

Notation. That means there shouldn't be anything new here, just a different way of representing the same ideas that she already was assumed to have understood from the preceeding weeks of class. And she did understand what was going on during the lectures. For the most part. 

In grad school, the core courses you take are basically the same core courses you take as an undergrad, or at least they start from the same place, the beginning, but in grad school they moved much faster and went into the depths of tedious calculations with long, dozens of pages of algebra that you skipped the first time. It turns out there are only so many good problems with nice looking solutions, and those are the ones you do as an undergrad. It also turns out that sometimes reality can be genuinely messy, and a solution that has ten different terms, none of which simplify, might still be the right answer. But not this time.

The point of the Young tableaux, like most of the tricks physicists love, is to make things look less complicated than they really were. The rules for combining boxes weren't derived using guess and check, there was deep mathematics behind them, group theory and symmetries. It took only about a page of algebra, and she had her answer.

{\begin{align*}
2\otimes 2\otimes 2 &= 2\oplus 2 \oplus 4 \\
%\Yvcentermath1
\yng(1)\otimes\yng(1)\otimes\yng(1) &=
\yng(2,1)\oplus\yng(2,1)\oplus\yng(3)
\end{align*}}

\mymark

It was already 1pm and Alice hadn't left her room yet. She hadn't been in her bed the whole morning, but she was there now, still in her pajamas. She had answered a homework question though, so this definitely counted as a productive morning. Actually there was a second part to that question, about expressing the right side of the equation in the basis of the left side, but still. Productive morning. It was time to get up and move around though. Since she and Elsa, well Elsa really, had automated that testing procedure, she wouldn't have to have to go into the lab until next Thursday, but she decided she would stop by    and unplug the power supply since the tests should be done by then. 

For sure nothing would happen if she didn't, but she couldn't really code and she wanted Adam to think that she was conscientious at least. And if the building did get hit by lightning and the boards were saved because they didn't have a direct line to the surge, that was the kind of anecdote the senior professors who didn't even know her would tell their professor friends, bragging about how well even their first year students had been trained. And if it happened the other way, where the building got hit and the boards were all fried because some first years had forgotten to unplug the power supply when they were done, that was a story too.

\mypause

It wasn't long before she was on her way toward campus, walking. Alice lived on the east side of the capital and it was about XXX miles between her apartment and the physics building, her other home. It was Friday, which meant that probably everybody else would be there too, as opposed to Saturday where you'd run into a third of their class or Sunday when half would be there at any given time but pretty much everyone cycled through at some point. 

There were 38 of them in the incoming class of 2007, two girls, her and Elsa. There were a lot of plasma physicists, some in condensed matter, quantum computing, phenominology, and there were a couple different groups for the handful of experimental high energy physicists among them. Aspiring, that is. Not there yet. Not by a long shot. 

Most people in high energy worked with IceCube, a neutrino experiment at the south pole whose principle investigator Francis Halzen was a professor at UW, but there were also groups for both of the big LHC experiments, CMS and ATLAS, and those who were interested could get involved with the supersymmetry phenominologists in producing simulations of proton-proton collisions where the laws of physics where whatever you decided they were.

\mypause

It was a little longer that way, but she decided to turn up the hill and head toward the capital. From there she could walk the whole length of State Street, and that was always nice. Corrupt as it may be, she had to admit the capital building was beautiful. It was modeled after the white house in DC, but had four great wings instead of two, each pointing in one of the cardinal directions. Like the white house, it was capped with a magestic dome and surrounded by an immaculate lawn, but unlike the white house, anyone was free to go on the lawns or enter the building. I'm sure they have good reasons, thought Alice, but it's too bad people can't just wander around the capital and see what the politicians are actually up to. Catch Obama walking by in the hallway chatting with a dignitary from Zimbabwe. 

The capital building sat at the highest point on capital hill, about 100 feet in elevation above the two lakes, Mendota and Monona, which flanked either side and defined the geography of the city. Madison is on an isthmus. That means that the city can't possibly expand in most directions and ends up making it all very walkable.

She had been planning on stopping by the carts for lunch on her way, but since she was already almost there, she may as well stop at Gotham Bagels instead. The shop was just off the near corner of the square surrounding the capital, and was the only place in the Midwest, Alice had decided, that made real bagels. Chicago probably had some, but she hadn't found any in Milwaukee the couple of times she had been, and definitely no others in Madison. Inside was busy, filled with young men and women in suits who appeared to be lawers or where anyway somehow involved with the capital or politics, but the line moved quickly and she got a super egg everything bagel, toasted, with the cream cheese of the week, chunks of Wisconsin sharp cheddar and chopped jalepenos. 

She ate on the capital lawn near a young mother with a picnic blanket and two young children, and made it to her office by three. 

\mypause

Since girls can only be officemates with girls, this meant the two of them, Alice and Elsa, shared an office intended for three people, but additionally since Elsa was never around, that meant Alice had the office to herself, which meant that she had plenty of free space, two extra desks and a chalkboard with the good kind of chalk that she stole from the math building, which meant her office became the natural hangout spot on the first floor for all of the first years, the grad student lounge being all the way up on the fourth floor, though it had a ping pong table and a microwave which her office did not, and since all conversations between first years eventually turn to discussing the homework, this meant Alice could sit at her desk and start working on the rest of her problem sets and eventually people would wander in and maybe help, which she did.

\mypause

"Making any progress?" It was Sam. He wanted to do plasma physics so instead of taking classical mechanics the first term like Alice had done, he had been in electromagnetism, but they took quantum 1 together and were now both in quantum 2 and stat mech. Alice was also taking particles 2 this term which, like the previous term, was taught by the theorist Michael J. Ramsey-Musolf who was well known in the small world of theoretical high energy physics, and off giving seminars or conference plenaries for about half of the classes, which he never missed, but gave his usual lecture via a weblink students could log in to from whatever quiet space he had found with an internet connection. He was also gay. 

"Some actually." Alice swiveled her chair toward the desk by the door as Sam came in and sat down. "For quantum I got $5$a this morning, and I have $1$, $2$ and a few ideas for $4$. Stat mech, we did question $1$ together the other day and I haven't touched it since." 

"That's not bad," said Sam. "I've been working on the stat mech and I think I have the answer to question $2$ but I need to really check it as I write it up. For quantum, are you saying that you get this Young tableaux stuff? Nice, can you show me? Hold on, let me grab Harry." 

Harry was Harold, a boy that had gone through high school and college and was now in his first year of grad school without ever growing a whisker of facial hair. He was a little shorter than average, and skinny, and had never been called anything other than Harold before coming to Madison. It was Ben that had first started calling him Harry, and since Harold smiled and almost half laughed when he realized Ben was talking to him, Ben kept doing it and before too long his name was Harry about half the time and Harold the other half, depending. Harry had never had an nickname before, and liked it, but he also liked it that his teachers still called him Harold. 

\mypause

Alice told them about the rules for calculating with the tableaux and showed them her solution and explained why the antisymmetry requirement meant there could only be three terms in the answer instead of four like you might expect. It didn't take long, and pretty soon they were all on the same page. It was nice to be the one giving the answers, good karma. And maybe just a touch of pride.

Since part b was just translating from one notation to another, they decided to finish question $5$ and moved across the hall where there was a bigger chalkboard. These kinds of problems usually went faster with a few people checking the algebra. 

It took them another two hours because there was a trick. They got most of the way there but the problem was that there are two $\yng(2,1)$ in the solution and they both want to identify the same pair of states. Since there are 8 unique states to begin with, there have to be 8 unique states to end with, not just 6, and the trick was to take linear combinations.

Daniel came up with it and popped his head in to let them know that he thought he had figured it out and was also working on $5$b in his office. Six out of eight with the last two from Daniel, that's not too bad, thought Alice. They copied down clean versions of what was up on the board and stared more or less blankly at question 3 for a few minutes before Alice decided to leave the two and walk downstairs to the lab. 

"Dinner?" asked Sam. 

"Sure," said Alice, "maybe in a half hour? I had a late lunch. Just let me run downstairs and unplug these power supplies real quick. Then we can talk about stat mech if you're sick of quantum." 

"Deal." 

"Harry, you in?" she said. Sam had a crush on her and she didn't want to encourage it. But he was a nice guy. All of the physicists she knew were nice guys really, Harold especially, and some were even decent looking. But she wasn't in to them. Sorry, nice guys. Actually that's not true. They were either nice guys or total pricks, but the total pricks didn't usually talk to girls so it was almost like they weren't even there. They were though. 

Harry said "Sure, thanks." 

\mypause

Adam was in the lab, strapped to the grounding bracelet, and she said hi when she walked in. Professor Olsen, 6pm on a friday? The installation is in 2013! "I was just coming by to check in on the boards and unplug the power supply since we're done with our tests."

"Ah, great. How do the results look then? Did the board pass the tests? And don't do that. You should unplug the boards from the power supply, not the power supply from the wall, I'm using it too. Actually, I just took that script you and Elsa wrote and modified it a bit. Very nice." He had said hi Alice back to her when she came in, but didn't look up and was focused instead on a little black square that bumped off of the circuit board and connected to the rest of the electronics via tiny metallic legs, like a robot twentyfouropede that was frozen in time. Or since it's a robot, maybe it was just powered off or recharging on its special platform. Or maybe it had gotten stuck in the La Brea solder pits, one of the many hazards a robot twentyfouropede may encounter as it wandered the circuit board plains of its natural habitat. 

"Actually I haven't looked at it yet but hold on." She pulled her computer out of her backpack and opened the terminal application and logged in to the computer sitting in the corner of the room, near the power supplies. She did this using ssh, which stands for secure shell and the command you run to log in from your computer to some different one is {\ttfamily ssh <username>@<computer name>:<location on the computer where you'll start your session>}. This was something she had learned on her first day working with Adam, and she felt a little like a computer hacker or a spy whenever she did it, even though this was basically the equivalent of opening the lid of her laptop, just on a different computer. {\ttfamily ssh alice\-@uwhep\_ps\_03\-.5226.wisc.edu:\-/home/alice}, entered her password, and she was in.

Each of the tests on the board ran separately, and made their own output files, so the last thing Elsa's script did before it finished running was to collect all of those files and bundle them into a tarball, a single file containing a compressed version of the test output files, in this case named {\ttfamily pcycle100x\_\-board06\_\-2008\_04\_29.tar.gz}. Elsa had done this kind of thing before. 

Alice had too now though, and she copied the tarball to her directory since it wasn't that big, just a few kilobites, and she didn't want to accidentally destroy their original files. Adam had told her early on that nobody ever remembered the flags you were supposed to use when extracting a tar.gz file, but the letters had stuck in her memory. {\ttfamily z} was for zip file, {\ttfamily x} for extract, {\ttfamily c} for compress, {\ttfamily v} for verbose mode so that it printed to the teminal everything it did, and {\ttfamily f} was for something. Force maybe? Either way, you used that one too, so the commands were {\ttfamily tar -zcvf output.tar.gz inputfile1 inputfile2 ..} to compress and  {\ttfamily tar -zxvf tarball.tar.gz} to extract, and Alice appreciated that all of the important flags were on the bottom left row of the keyboard, all in a row. Still, she had a webpage with the flags bookmarked, and a text file with the tar/untar and ssh commands saved on her desktop.

\mypause

Adam came over while she was trying to figure out how she could possibly look at the hundreds of logs that had been generated, bracelet off and hanging from an improvised wire hook twisted onto the rack in the middle of the room. Now she had his full attention and they checked a few random logs from a range of different power cycles, including the final one, manually ran a few tests that weren't included in the script, and concluded that the board had likely passed with flying colors but that they should write a script to go through the logs and make sure. "This means, of course, that we have a procedure now. So the next step is to test the rest of the boards like this." 

Bah. Of course. Of course that was the next step, to test the rest of them and not just this one. "Of course," Alice agreed. So she set up the next board, opened the script and modified line 5 to read {\ttfamily outname =} {\ttfamily  'pcycle100x\_\-board01\_\-2008\_04\_30.tar.gz'} and started it running.

Adam began packing up to leave and so did Alice. "Thanks for getting this going today, I think the boards are looking good," he said, stopping in the doorway. "And also, can you make sure Elsa knows not to unplug the supplies? Thanks, ok, have a good weekend."

"Sure, no problem, you too." And a few seconds later she heard the heavy door to the staircase slam itself closed and she was alone in the lab. It was kind of exciting being alone in the lab with all of these fragile and expensive electronics. A little scary too. If she was drinking water and tripped and splashed it in the right spot, she could probably cause a few tens of thousands of dollars in damage. That would be a story too, and she wasn't going to let it be one about her. And anyways she had already been down there much longer than a half hour as she had promised.

\mypause

Upstairs, there were now five of them working together, and Daniel had had an inspiration for question 3 on the quantum homework. Question 3 was the kind of problem that was tough to set up and then tough to solve, but it was also straight forward in a kind of way. What was it Einstein had said? That the problem he was working on wasn't difficult, it was just tough. There weren't tricks, except for math tricks you had to use in the pages and pages of algebra, and it was more or less a direct application of setting up and solving the equations to describe an electron traveling through a complicated electric field. 

Brute force with no way out. Question 5 had been subtle, it was about looking at concepts from different angles. Question 3 was about taking a concept and seeing what implications it had on real particles doing real things in the real world. 

Maybe not exactly the real world. These were the equations to describe what would happen in a universe containing just this one electron and just this one electric field, but the important thing was that the laws of physics in this almost empty universe were hopefully the same as the laws of physics in the real one, it just didn't have as much stuff in it that might complicate things. 

They were all hungry by now though, and Sam promised to email Alice the pictures he had taken of the chalkboard as they worked through the integrals in the other room. "Thanks Sam."

\mypause

They ended up at Doty's, which had good hamburgers and great cheese curds, a midwestern delacacy that was new to Alice when she first arrived in Madison. The closest thing to cheese curds they have in Pennslyvania are maybe mozzarella sticks, battered and deep fried cheese, but instead of mozzarella, these were unpressed cheddar curds that came with a variety of sauces, golden brown, hot, and covered your fingers with grease. Doty's also had a good collection of beers on tap, another thing Alice had quickly learned Madison is known for, and exceedingly proud of.

After the meal and a beer or two each, everyone agreed that they were getting tired and were ready to start heading back home. Lame, but all the better, thought Alice. She was tired too, or maybe just drained. She still had energy but was getting tired of thinking. It had been a solid physics day, and she also wasn't particularly interested in being pulled into a night out with a group of exclusively male physics graduate students.

Ok, she had done it before, and it wasn't that bad, and she did like these guys, even Sam had his moments, but not tonight. She had finally convinced Lesley to go out with her to Plan B and tonight was the night. She hadn't been there before, but had heard it mentioned a few times as the only place to really dance in town, usually by cool Madison hipsters, but hipsters aren't always wrong. It was a gay club, and that meant you wouldn't have guys constantly moving in on you, but who knew what the girls there would be like. Neither Alice or Lesley had ever been to a gay club before, but they did both love to dance.

\mymark

"Hey girl" called out Lesley from the open door to her bedroom when Alice made it back to their apartment. "Welcome back, what have you been up to today?"

"Ug, physics" said Alice, setting down her backpack then falling into the couch. "And computers, and then physics, and then dinner with physicists where we talked about physics." She closed her eyes and sank a little deeper into the cushon. "Great day. What have you been up to?"

Lesley came out of her room but Alice's eyes were still closed. "Well. I checked on our seedlings, they're doing great, weeded the rest of the plot, and then I was catching dragonflies most of the day. And then I came back here and made some dinner."

"I did eat a bagel on the capitol lawn today" said Alice, opening an eye and rolling her head toward her housemate. "That's outside." Lesley was at the kitchen counter, opening a cabinet. 

"What you could use" she said, pulling down her bottle of Wild Turkey burbon whisky, "is a drink." Alice closed her eyes again.

"Ten minutes." She heard ice hit glass and some water run, then the freezer door close. Pouring into two glasses.

"Now. This was your idea, remember?" Lesley sat down on the chair and Alice sat up on the couch, static electricity pulling the hair on the left side of her head over and across her face. 

"Ok, I'm up. I did already have a beer at Doty's though" said Alice as she reached for the glass and pushed her hair back, taking a sip. Oof, Lesley was the whiskey person, not her. It wasn't that bad though, once you got used to it, and the scottish bar on the capitol square was a place the other grad students talked about, so she had a good excuse for getting used to whiskey. "So then how was the hunt?" Lesley was a biology grad student and her research subject was dragonflies. 

\mypause 

Dragonflies are simple. They're about as simple as you can get, short gestation period, small number of neurons, big eyes, tiny brains. They have four wings and can catch small insects in midflight, but nobody understands how they do it. That's the mystery. Part of their control has to do with the structure of their wing joints, but how they go from seeing a fly in it's randomized flightpath to figuring out how catch it is almost a complete mystery. 

At the end of the day, you can probably think of neurons like a chemical version of a computer. Since dragonflies are so simple, scientists think that the algorithms running on their neuron computers are probably simple too. Simple in the context of neurons that is, which are incredibly complex even before you consider what happens once they start interacting with each other. 

But if we could understand what those algorithms looked like, then that would mean all kinds of things. Even if we could only understand parts, it could revolutionize our understanding of neurons and brains, not to mention robotics and also dragonflies. It was interesting research, and important too, but Lesley found it a little weird that the grant she worked under came from the Department of Defense.

It was a good hunt. In total, she caught XXXXX TALK TO ROB OLBERG ABOUT WHAT A GOOD DAYS CATCH IS AND IDENTIFY WITH SCIENTIFIC NAMES (gender?) XXXXXX, out in the wetland area behind XXX REAL PLACE OUTSIDE MADISON XXXX. XXX dragonflies. Too good, because now she also would be in a lab when she wasn't in class. Her job was to put tiny electrodes into their heads and take super slow motion videos of the specemin as it tried to catch flies. Out in the wild they were dragonflies. Once you caught them and brought them back to the lab, they were specemins. 

\mypause

They sipped on their drinks as they got dressed in their rooms, chatting about what they thought it was going to be like in a gay dance club. "Do you think girls will hit on us?" Lesley asked. "Or guys?" 

"I don't know" Alice responded, "maybe, I bet we're going to see some good dancers though."

"Oh yea, for sure." They decided that they didn't want to get there too early, and would start walking over at 11. 




%Goes to terrace, too cold for anyone else. Just her and her good friend Jae Jook Sakurai. J. J. Sakurai was a physicist. He died in 1982 but his textbook Modern Quantum Mechanics lives on. 


