%\documentclass{book}
%
%\begin{document}
%{\Large Professor}\\

\chapter*{Adam}

Peter Higgs is the one with the name on the boson because Higgs is a monosylabic name. Because Higgs has only one syllable. Because he's the only one who deserved it. Just kidding. Peter Higgs is the one with his name on the boson because his name easy to say.

Professor Adam Olsen liked that last one. The Anderson-Brout-Englert-Higgs mechanism if you put their names alphabetically, correctly, but Adam preferred to think of it as the Anderson-Higgs-Brout-Englert mechanism. Higgs himself referred to it as the Anderson-Brout-Englert-Guralnik-Hagen-Higgs-Kibble-'t Hooft mechanism, the Dutch Nobel prize winner Gerardus 't Hooft being last on this list, also alphabetical. Gerardus is always cited as 't Hooft, getting 1.1 names when everyone else gets one, but he is a genious by all accounts. 

Tomorrow Adam was giving a lecture at the Monona County Library in their new auditorium, a talk titled One Last Piece: The Serach for the Higgs Boson, and he wanted a good opening line. A little clever, nothing edgy, and definitely nothing too sciency. If there was one thing he had learned about giving talks to the public it was that you could never go too shallow. 

``They're not there to learn," he had told his wife Kara some weeks ago, when they were in bed that evening after receiving an email earlier in the day from Sarah at the Monona County Library Alumni Association inviting him to participate in the Monona County Distinguished Lecturer series. ``They're looking for a few quotes they can show off to their friends and the chance to remind everyone of how they wasted \$500,000 on an acoustically pitched auditorium in a library. In a library." He repeated, starting with vigor but his voice losing some of it's conviction by the end as he realized he wasn't getting the response he wanted. He turned his head towards hers, ``In a library?" 

He loved his wife and she was giving him that look. ``Still. They have two great auditoriums at the University, and the chapel, and the lecture halls. And you know all of their Distinguished Lecturers are professors at the University anyways. And how do you become an alumni of a library?" 

``Dear," she took that tone of voice, she was a mother after all, ``not everybody wants to spend their evenings in a lecture hall." 

\mypause
%\vspace{5pt}

She was right, and he had to admit that it was nice to do things off campus. But the auditorium was still a waste of money, and these alumni weren't going to understand half of what he was talking about regardless of what he said. Where should he start? He had plenty of slides from other lectures. He could show slides right? The LHC collided protons. Could he assume everyone had heard of a proton? These charged particles sitting beside their uncharged twins in the nucleus of an atom? These charged particles whose mass was conveniently almost exactly one in the units particle physicists preferred? These particles whose charge and mass defined how an atom behaved? What it was? Maybe twin was the wrong word. Sibling at least, closer than cousin. Protons and neutrons had almost the same mass and were made from different combinations of the same quarks: up-up-down for a proton and down-down-up for a neutron. Nobody in the audience would know that. So basic a fact, that everything we see is made from atoms, which are made from protons, neutrons, and electrons, and that protons and neutrons are further made from smaller particles called quarks, that come in six different types. Nobody in the audience would know that.

 Ok technically everything we see isn't atoms, it's light, or maybe neurons firing somewhere depending on where you defined the beginning of vision in that long chain of reactions and interactions, but he wasn't going to bring that up with the audience either. Still, everyone had heard of a proton, right? Probably not starving Africans, but they wouldn't be there tomorrow anyways. What about Indians? What did they learn about in school? He decided he would ask his friend Mohammed, a condensed matter physicist, the next time they ran into each other.

 His thoughts wandered to Sarah. Who was she? And how did she get his name? Did she care about particle physics at all or had she just heard the name God Particle and decided that was enough? She did at least use correctly the acronyms CERN and the LHC, the first one being the Centre Europeen du Récherche Nucleaire, the site and orgainzation which housed and funded the second of the two, the Large Hadron Collider, a proton-proton particle accelerator and collider, the largest particle accelerator ever built, which was set to turn on in the late summer or early fall and would, among other things, provide Adam with the data that would be necessary to secure his tenure.

Adam Olsen was housed and funded by The University of Wisconsin-Madison, or funded at least, and that paid for the house. He was an Associate Professor with a grant from the NSF, the National Science Foundation, and could afford a postdoc, two graduate students, two conferences a year for himself and one for each of the graduate students, and an undergraduate summer student that would probably start some time in June, after finals got out. Hopefully they wouldn't try to take two weeks right at the beginning like the last summer student he was warned about. The grant he was on was part of a larger grant held by the senior professors at three universities, his own senior professor being Joseph Goldwater Professor Steven Jacobs, that primarily included upgrades to the subsystem of the particle detector he worked with which were slated for installation in 2013 pending the results of various ongoing studies and, of course, the budget. 

Professor Jacobs's birth certificate claimed he was named Steven Isaac Jacobs, 6 February 1939, and preseumably he was in part funded by Joseph Goldwater. His house he had inherited from his grandfather, who had himself helped in the construction to keep the costs down and to watch the Mexicans and left the house to Steven as the only one of eight grandchildren who still lived in Wisconsin when he died.

\mymark

Kara was up first, around 6:20 as usual and she was energized. She glanced at the sleeping Adam, his flickering eyelids, and closed her eyes again, imagining herself as a new species of inchworm that inched sideways toward the edge of the bed. Left hand, left foot on the ground, she slipped from under the covers and used her strength to hold a sideways handstand for a moment before quietly setting her other hand and foot on the ground in a downward dog. She held it for a few seconds and breathed slowly before working her hands back to her feet. Another pause. Breaths. And she slowly opened her eyes, rolling up and feeling the tension in the back of her legs give way to a clenching, then relaxing, each muscle one by one as she reached toward the ceiling.

She glanced at Adam, undisturbed, and swung her arms to get the blood flowing as she pulled on some shorts and a sports bra. Out the bedroom and she peeked in on the twins, Chloe and Walter, still asleep too, in their separate beds. Separate but matching. They had gotten them sheets, flowers for Chloe and mountains for Walter, but Walter insisted that he wanted flowers too, so now they were the same and sometimes she found that they had switched beds during the night. She closed the door. Nobody could touch her. Down the stairs, shoes, and she was turning left on Elm Street before she realized that she had forgotten her hair tie.

Aha! There was a band on her wrist. Clever clever girl. You are such a clever girl. Clever girl, clever clever girl, not insane at all, you just like singing to your self, I just like singing yourself, myself, yourself, myself, such a clever girl. She was in her stride. And she loved mornings. Usually. Sometimes on the weekends she would sleep until ten and that was amazing, but she loved this morning at least. The trick was to get out of bed the moment you thought you might be awake, and out the door before your body realized it wasn't still dreaming and the muscle contractions were in fact real and it hadn't been doing this kind of thing forever but just moments ago was completely still, still unaware that she was awake.

She was fast. She told people that she used to be fast, back in college at Loyola where she ran the 10k for the track team, but she knew that she was still pretty fast, especially if you thought about how many people didn't run at all. She was pretty fast for a runner and runners are a small minority of the general population so she was probably one of the fastest people in the world. In some percentile at least. Maybe 10th? 20th? The whole country of Kenya was probably faster than her, different kind of minority. She thought about the overlap between runners and morning people. It seemed to be pretty high. Or whatever's the adjective. What is that adjective? High overlap. Lots of overlap. High overlap factor. Nerd.

\mypause

By the time she got back to the house, Kara had been running for an hour and Adam was up and cooking scrambled eggs while the twins hopped around in the living room, having invented a game which seemed to involve hopping on one foot and occasionally pulling the other's hair. Good timing, she thought. Deliberate timing. She knew that Adam knew that she would wake up on her own around 6:20, run for an hour, or at least make sure to be back by 7:20, and that he felt a little guilty about not getting up with her, though not enough to lose precious sleep over, and had woken up at 7:10 or maybe even 7:15 and quickly got dressed, washed his face, woke the kids, who were already awake, and cracked the eggs before she came back through the door. Good man. And he could be efficient when he wanted to. But he was a creature of the night. It wasn't that he didn't appreciate mornings, but they were always tomorrow and the night was filled with secrets. Quite possibly the only reason he wasn't still asleep was because he had learned how to get up immediately, a trick he had learned from his wife.

\mymark

After breakfast, she took a shower while he walked the kids to kindergarten and they rode their bikes down XXXX street, right on DDDDDD, past the Bucky the Badger Stadium or whatever it was called, and on to the university campus where she turned left to get to the biology and neuroscience building and he turned right. It didn't really matter what time either of them got in since they each taught only one class in the afternoon, but she liked to get in before nine to set an example for the graduate students who would come in at noon if it weren't for her presence, and he appreciated being in early enough to catch the afternoon meetings at CERN, which were usually physically located on the Meyrin site, outside Geneva Switzerland on the boarder with France and seven hours ahead of Madison Wisconsin, but which were also always simultaneously broadcast online using EVO. He liked to put the meetings on in the background as he worked in the lab, like a carpenter putting on sports talk radio in the shop as the commentators argued and debated over last night's game and what it meant for the season and the clever tactics the defense had used and that rookie who was making waves and strategy, strategy, strategy.

%depending too much on secretary? she chose name of talk

\mymark

It was just after lunch when Adam ran into Mohammed. He was still hanging around in the physics office, chatting with the department secretary, Andrea, who had been scolding him for his lunch selection of two peanutbutter and jelly sandwiches and a cup of water. "And what title did you end up going with for the talk?" she asked.

He took a breath, still embarassed by the name. "One Last Piece: The Serach for the Higgs Boson." It was Andrea who had come up with the title. She grinned when he told her, and he had to admit that it was kind of catchy. Sounded like a mystery novel. Some hokey tale where everything was dark and sinister and the hero always wore a cloak and hid his face, but the plot kept moving and it wasn't the maid after all but her sister who was just trying to protect her in the end. Not a great book, but entertaining at least. That was all he should really hope for tonight, try to be entertaining and to finish on time. He'd be giving the alumni their quotes, but at least he could choose which quotes to give them. Physics can be hard to reduce to sound bites. What if he just gave a lecture, a real lecture tonight? What if he just dove right in and talked about the problems, the real problems, in the field these days and the latest theories and crazy ideas people were having trying to solve them? At least he probably wouldn't be asked to do another one of these. 

Mohammed came in looking for some chocolate covered pretzels that Andrea had put in a bowl on her desk. She always had the best snacks and took care of her poor physicists who would probably lose their heads if they weren't screwed on tight or at least starve. It was important to keep their blood sugar up and coffee could never be more than three minutes out of reach. As they were walking out, heading towrds the stairs to the basement where their labs were, Adam remembered his question from last night. "Do people in India know about protons?"

"What do you mean?" asked Mohammed.

"Like if I asked a random person off the street if they knew what a proton was, would they say something true?"

"In India, that depends quite a bit on the street," Mohammed replied, "and the person I suppose."

"What about children, do they learn about protons in school?" Adam asked.

"Do American children learn about protons?"

"I don't honestly remember. I think so, eventually."

"Then so do Indian children."

"Why do you say that?"

"Because Indian children learn what American children learn. They watch American children and they learn from them. They imitate, but only sometimes or only on the surface, while the American children are the stars in the movies and the American children's lives are the movies."

"They do this from India?"

"From India." 

Adam wasn't sure how to respond, but he sensed truth in his friends words, and many layers of meaning, not all of which he was sure wanted to talk about then and there in that basement hallway of the physics building. "You come from Bollywood, right?"

% Dighati
"Yes, I come from Bombay. And it could be said Bombay is near Bollywood," he smiled. Mohammed was always a good sport. "American movies could use more singing."

\mymark

Adam's lab was nothing like Kara's. In her lab everything was glass and everyone wore white coats and latex gloves and disposable plastic goggles. They worked around ventilated hoods and held beakers with tongs, or better yet, fixed them in place and moved their contents with pumps and evaporators. Only the exit sign and designated evacuation pathway glowed in the dark, but Kara's lab looked like it should be handling chemicals that flouresced neon green or a ripe, dark purple, and she noticed a change in her students behavior as they stepped into the lab. It was the same as how the military have their uniforms, or medical schools have a big ceremony the day their doctors-to-be get the famous white doctor coat with their name embroidered above the breast pocket, just large enough to hold a small notebook. Or how you dress for the job you want and not the job you have.

In Adam's lab, there were no lab coats. No goggles or gloves, but he did insist firmly that if anyone ever touched any electrionics, they must be wearing a small elastic bracelet with metalic interweavings that had a snap connection to what looked like a telephone cord, the far end of which was stripped to bare wire and wrapped a few times around the metal rack in the middle of the room. "As you're moving around the lab," he explained on the first day and many days since, "you rub against things, even air, and some of your electrons can get knocked off. Or you might pick up a few, especially if you've been near the power supplies. But the point is that since your shoes are rubber, you're electrically isolated from the planet and you can pick up a slight relative charge. If there's an easy enough path for this charge imbalance to dissapate, you'll get a discharge, like when you walk around with wool socks and shock someone. But with our circuit boards, this can happen at a much smaller scale and you won't even feel it. So always ground yourself or I will ground you." He showed just a trace of a smile as he thought for a moment about himself working on his boards, with a hole cut through the floor as if to let the earth below the basement see what had become of this space they had annexed, shoes off and buried up to his knees in rich dark soil, unable to move from his station but finally unteathered from that annoying bracelet that always pulled on your arm at least just slightly.

They were working on an upgrade to the Cathode Strip Chambers. CSCs. The CSCs were a subsystem which, along with the Drift Tubes, DTs, and Resistive Plate Chambers, RPCs, formed the Muon System, which was one subdetector along with the Tracker, Electromagnetic Calorimeter, ECAL, Hadronic Calorimeter, HCAL, and of course the 3.8 Tesla superconducting magnet, that made up the Compact Muon Solenoid, CMS, detector, which joined A Large Toroidal LHC ApparatuS, dumb name and a cheap ploy to show up first in publications as ATLAS, A LHC Ion Collision Experiment, ALICE, also not a great name but not as bad as ATLAS, and the modest LHCb, as one of four megasized particle detectors stationed at collision points around the LHC. Actually their upgrade was only to go on the CMS endcaps, as opposed to the barrel, the detector being shaped generally as a series of concentric tubes centered around the beampipe where protons entered the machine from both ends and collided, with two massive endcap discs to ensure that as many particles as possible would pass through some of the detector volume and not just escape out one side or the other.

2013 was still years away, five years away, but already this upgrade had been years in the planning. The goal would be to install their contribution to the CMS detector during LS1, Long Shutdown 1, after the LHC had finished with a few years of shattering previously held world records for collision energy and luminosity, and hopefully producing enough Higgs bosons to be clearly seen above the other backgrounds. That was Adam's concern. Not that the Higgs boson wasn't there, everyone agreed that it had to be there, but that the backgrounds would be too large for a clear signal. 

In truth, Adam sometimes felt as though had very little to do with the Higgs search and he was ok with that. He was trained as a physicist but had ended up closer to an engineer, building and testing hardware for the experiment. The people who would find the Higgs would be analyzing the data. They would be grad students and postdocs and even a handful of professors writing code to buid their own analysis software, up late and drinking coffee, silent except for the clickling of keyboards in a small room of five twenty somethings, each absorbed by their screen, and sharing a bond none but them would ever understand. It would be romantic. It will be called a triumph of human intellect. The title One Last Piece really wasn't that terrible. The Higgs boson was the one last piece. The last piece of the standard model, the missing particle that explained everything, that made it all work, that was almost too good to be true. It was the last genuinely new particle to be predicted in the past thirty years that anyone believed in, and everyone did their calculations just assuming the Higgs existed, putting appropriate uncertainties on the values of the parameters it was known to have.

The LHC would find the Higgs, and everyone would be able to breath a sigh of releif, but nailing down parameters like it's mass and how strongly it interacted with other known particles, like quarks, was the real goal. That and supersymmetry. If they discovered supersymmetry, the Nobel would be a guarantee. How would they award it? Would they really give a Nobel prize to the 10,000 authors co-signing that historic first paper?

\mypause

It was a quiet day at the lab and the students, both in their first year and both at least reasonably hard working, were trying to automate a procedure for testing one of the boards. "Hi Adam," said Alice as he walked through the door. His students always called him by his first name, just like he had always called his advisor by his first name, Gary. "What do you know about the tunneling protocols for communication with these power supplies?" Direct, he appreciated that. Students should run into problems, get stuck, and work on them for a few hours. But only a few hours, there's no point in wasting time with questions someone already has the answer to. 

Alice reminded him the model number of the power supply and he found the users manuel, a thin paperback, sitting neatly among two dozen or so other manuels of various heights and widths, most of which covered in dust, on a metal bookcase that looked as though it had only long ago been thought of as temporary. They found the protocols, the three of them, Adam, Alice and Elsa, and Alice and Elsa spent the rest of the afternoon writing a script that would turn on the power supply, run a subset of the tests they had been performing on the board the supply was powering to make sure the board was functioning properly, power off the supply, and repeat. Previously they had had to flip a switch up to turn on the supply and then back down to turn it off when they were done, every single time. This wasn't so bad when they were first testing their boards, when most of the time was spent running the tests, but the boards were looking good and now they were checking to see how they would handle the artificial aging process of turning them on and off over and over.

The tests take 15 minutes to run and they wanted to age them with at least 100 power cycles. 15 minutes times 100 cycles equals 1500 minutes, divide by 60 gives 25 hours. They could either sit next to a power supply flipping a switch every 15 minutes for a day straight, or they could write what ended up being 24 lines of code to do it for them. To be fair, those 24 lines called many many more lines of code, most of it hidden in libraries which imported secret commands designed to unlock functionalities some unknown programmer had found useful and you might too, but the final script they had by the end of the day only contained 24 lines of code. Some students, he knew, would choose to flip the switch manually, and sometimes they would even be faster for having done so.

As the girls worked on the code, Adam worked on his presentation. Students. They weren't his girls, they were his students. He made a conscious effort not to use gendered pronouns, in part he admitted to himself, because he was never sure if he should use girl or woman when adressing a lady. Not lady, definitely that was a bad one to use. Elsa was the one who actually wrote the code, but Alice seemed involved and mostly stood, pacing and asking questions or pointing out syntax errors. One Last Piece. Definitely tacky. 

\mypause

Adam came home directly after class. He was teaching Introduction to Mechanics for undergraduate nonphysics majors who feared calculus. Physics majors and engineers had their own sections, but this class had 600 students, 300 per section which met twice a week for 50 minutes and also in 20 mods of 30 students where they met with the teaching assistants and went over problem sets and problem solving techniques. Sometimes he envied the TAs who got to really know and interact with the undergrads, but he also enjoyed having enough time to actually focus on his research and not just grade papers or explain Newston's second law all day.

Kara's class ended at 3:25 and she had apparantly left the lab early since was already home and had picked up the twins who were now playing a new game that was completely different from what they were doing that morning because this time they were hopping on the other foot and only pulling the hairs right next to the hairs they had pulled before because they could tell and all hairs are different and they had perfect memories that could remember anything they wanted it was just that they didn't want to remember everything because then they would fill up and explode and so sometimes they forgot to eat asparagus or not watch TV but it wasn't their fault, they just needed space to remember important things, like that FACTABOUTABUG. "Good point, and I didn't know that about the BUG. But you should remember to eat your asparagus too."

It was nice when they were all home before 6, but they were at least almost always home before 8, unless there was a real crisis at the lab, either lab. They were becoming better and better friends with that girl across the street, Molly, who was in 10th grade and would sometimes pick up and watch the twins for them after school got out and had a cell phone. Crisis can be a nebulous term. And Molly was a girl, not a woman. Responsible though.

\mymark

It took almost an hour to bike the four miles to the library, but they arrived with plenty of time. Madison was good like that, good for bikers. Walter and Chloe both had tricycles and invincible helmets and rode between Adam and Kara, all in single file unless the twins wanted to talk to each other. Adam told them they were on a skinny boat going through whirlpool infested waters, like Odysseus. He told them they were on the razor edge of a mountain chain searching for lost treasure in New Zealand. Kara told them from behind that if they didn't ride in line they were going to get hit by a car, or more likely, a bike coming in the other direction. Someone had to.

The library had finally shed it's scaffolding and revealed a new wing, built where had used to stand a convenience store and another front which had housed a bar, an italian restaruant, and most recently a taco shop before all going under and leaving the vacant windows boarded up. The new wing was all glass and wood, with only minimal amounts of visible metal and tastefully there, in places of support and framing the large glass panes. The waiting area was connected to the main lobby by a pair of heavy glass doors, and inside was carpeted, with comfortable chairs around the edges and standing tables with tasty looking ordervs in small clusters in the middle of the room. Hors d'oevures, whatever, thought Adam. Adam was american and he knew he didn't speak french and that's ok. So he called them ordervs and his pronunciation sounded just as good as anyone elses. 

The auditorium itself was all wood inside, with cushioned folding chairs fixed in arcing rows and dim lighting and room for 250 people. There was a small stage in the front, with a podium set up and blank white screen pulled down behind which would be perfect for watching movies or physics presentations. In the wall, on the outside next to the door, was a plaque announcing that this building was LEED Silver certified for being ecologically friendly and cited nearby towns from which 80\% of all materials had been sourced. 

\mypause

Sarah found Adam near the doorway looking at the plaque and told him that the building was LEED certified and that 80\% of all materials used in the construction had been sourced locally. She told him that he and his wife, that's her isn't it, the one with the adorable twin daughters by the hors d'oevures, should follow her to the bar because there were some people she wanted them to meet. He hadn't realized there was a bar. "I hadn't realized there was a bar."

"Oh yes," Sarah replied, happy to be getting at least something from this conversation with a physicist. She knew they had a reputation for being difficult to talk to and antisocial. "One of the goals of the Judith and Harold M. Butler Foyer and Auditorium is to provide a social space. A public space where citizens can mingle and interact. A place for intellectuals like you to enlighten us to the mysteries of the universe."  

"Ha," he laughed. "We'll see how much more light I can shed than what's already .. coming .." He was trying to make a joke about the ecologically friendly skylights and enlightening or shedding light on things. "Sorry, I was trying to make a joke about the ecologically friendly skylights and enlightening or shedding light on things." 

"Ha." Poor guy. Beautiful wife though. He must be smart. Actually he wasn't that bad looking himself. "Oh look, there they are," Sarah said, pointing as if she had just noticed, "there's Trisha and Michael Sampson over there by the bar. Let's go say hello to them."

Adam agreed, thankful to be leaving the one-on-one with this woman. La tête-à-tête en francais. See, he did know some french. She was trying to be nice though, in her own way.

He finally caught his wife's eye as they headed toward the bar and the Sampsons, and she introduced herself and the kids, Walter and Chloe, one boy and one girl though yes they are twins, and yes they are adorable, though sometimes a handful. Trisha agreed that children can be quite a handful but added that they were also the most rewarding thing in the universe. Everyone agreed to that and Kara suggested that Adam might want to check that his slides were being projected correctly and the situation with the microphone. "Good thinking" said Adam and he assured the Sampsons that it was a pleasure meeting them and that he would enjoy speaking with them more if there was time after the presentation. 

\mymark

The talk went more or less as expected. Adam talked about Newton, who everyone had heard of. He talked about Einstein and saw a few of the alumni nod their heads and whisper to their neighbors when he mentioned that $E=mc^2$ is just the equation for the energy of a particle at rest and that the full equation for the energy of a particle of mass $m$ and momentum $p$ is $E^2 = (mc^2)^2 + (pc)^2$. "Of course," he added, "particle physicists like to work in units where the speed of light, $c$, is one, so the equation becomes just $E^2=m^2+p^2$." Why did he say of course there?

"You can set the speed of light equal to one because units are arbitrary. Usually people think of the speed of light as $3\times 10^8$ meters per second, meaning a photon, a particle of light, travels three hundred million meters in every second. But what's so special about a meter? Back in HISTORYOFMETERBAR, and the metal bar is sitting as a reference out in Paris somewhere.

But what if instead of the meter, those parisians had come up with a different unit of length, the supermeter, whose length was $3\times 10^8$ meters, or better yet, what if they had simply defined the supermeter as the distance light travels in one second? Then the speed of light would be one supermeter per second, and that's basically what particle physicists are doing when they set $c=1$. It's a trick, and there's really nothing deep going on here, but it's notationally convenient."

He was getting off topic. Who cares that particle physicists had figured out a trick for getting rid of $c$ from their equations? He had meant to tell them that at the LHC, protons would soon be colliding at higher energy than ever before, and that $E=mc^2$ was why this created new particles. Get enough $E$, energy, on one side of the equation and the equals sign means that it can just as well be interpreted as mass, $m$, on the other side, a fact which the universe indifferently took advantage of. Mass means particles, so $E=mc^2$ means you can take energy and create new particles like they do at CERN. Or it means that you can destroy particles and create energy like they did in Hiroshima and Nagasaki.

He told them this too, but left out the part about bombing Japan. No need for that. He also didn't tell them that the nuclear bombs which were dropped on Hiroshima and Nagasaki were pitifully small compared to modern warheads, which use a full nuclear explosion just as the fuse to detenate the real bomb. Facts. You could like them or hate them, but they were true statements and you can't argue with them. Like the fact that when the Enola Gay AND NAGASAKI?? dropped their payloads on DATEDATEDATE, warfare came to a new era and could never return to knights and seiges. These are facts he knew, that probably every particle physicist knew, and had nothing to do with his research or what was going on at CERN. Not literally nothing, since they used some of the same equations and maybe even occasionally got stuck on similar electronics problems coordinating precision timing, but basically nothing. 

He told them about the periodic table and most people recognized it. He told them that particle physicists also had a periodic table, much smaller, for the fundamental particles, and that same handful as before whispered to their neighbors. Some people can't help themselves. He told them about particles and about fields and how particles and fields are really the same thing and just two sides of the same coin. He told them about interactions. He told them about scattering experiments. He told them about le Centre Europeen du Récherche Nuclaire and about the Large Hadron Collider. And finally he told them about the Higgs boson.

\mypause

"So we're looking for the Higgs particle, but really we want the particle because finding the particle means we've found the field, and the Higgs field is the important thing. You might have heard it called the God Particle. Please don't call it that." The audience laughed, awake again and paying attention to every word since he said Higgs. "They call it the God Particle because we say it gives all particles mass.

Think of it like this: the Higgs field is everywhere, filling all of space. Some particles interact with it a lot, like a bottom quark, and some particles don't, like a photon. For a photon, cruising through this Higgs field is like you or me running on the street. For a bottom quark, it's like instead of air, you're running through water. For the same amount of effort, you'll go slower the more you interact with the Higgs field.

So then what is mass? Newton's second law of motion states that $F=ma$, force equals mass times acceleration. Usually we use Newton's second law to answer the question if I apply $F$ amount of force to some mass $m$, how much will it accelerate. But you can also flip this equation around and say $m=F/a$, which defines mass as the ratio of how much force you apply to how much the object accelerates.

When you look at it like this, you can see that the more the particle interacts with the Higgs field, the less it will accelerate for the same amount of force. A small number in the denominator means a big number overall, so we say that particles that interact strongly with the Higgs field have high mass, and when we're being sloppy we just say that the Higgs field gives particles their mass." 

It was a decent analogy, and at least most of the audience was still looking at him and not down at their phones. It wasn't perfect though, and he probably shouldn't have walked them through the algebra. Either they got it or they didn't. Either they cared or they didn't. He wished he had a better analogy, that he could explain the elegance of the mathematics that predicted the need for a Higgs to someone without at least a bachelors degree in math or physics. To explain how the Higgs first solved the problem of massive gague vector bosons before demonstrating it's worth in the quark and lepton sectors, eventually becoming the crown jewel of the standard model. How it's perfect symmetry was universal and it might even give an explanation for inflation, that infinitessimally short period right after the big bang when the universe dramatically changed length scales. How the parameters associated with it appeared casually mixed in with the coefficients for other particles, as though it hadn't been a breakthrough to put them there, but that they had been there all along until Higgs one day happened to notice. 

He then breifly talked about his own research and the importance of the CSC endcap upgrade, but mostly just wrapped it up. After hearing about the Higgs everyone was ready to leave but he wanted to have a few slides he could skip over devoted to his own research for the sake of his funding agents who he would send this and other notable presentations he had made during the course of the year to via a large package of documents, compiled by Steven Jacobs, and consisting of his own research, which was admittadly minimal these days, as well as that of everyone under his grant, which Professor Jacobs would send some time in December. 

\mypause

A man in the audience with glasses held a portable microphone and asked, "This might be a dumb question but you explained the difference between the Higgs field and the Higgs particle. What's the Higgs boson then? Is that like a combination of the two? Thank you." And he sat back down.

"Good question. Not a dumb question, and I hope I didn't confuse anyone else with that. A boson is a type of particle, it's the name we give to a whole class of particles, named after the physicist XXXX Bose. The other class is called fermions, named after Enrico Fermi." Should he start talking about the uncertainty principle? Not right now, maybe if they follow up.

Another audience member stood up, possibly a college student with a good sense of fashion and a blue sweater. He asked, "I've read in a few different sources that many scientists are concerned about the LHC. They fear that you'll create a black hole and destroy the planet. My question is this: even if the chance is very small, shouldn't we consider destroying the planet as an unacceptable risk?"

Little punk, probably wasn't even 25. "Are these many scientits physicists?" He didn't give him time to respond, probably they were, some of them. But he couldn't help himself either sometimes. "You're right though, that if there was a chance we would destroy the planet, that would be an unacceptable risk. The fact is that we won't." He kept a smile but could feel his face tightening a bit. Like when people insisted in calling it the God Particle, even after he had given them the running analogy, and they said that the name still did seem appropriate. "The design energy of the LHC is about 7 times higher than the highest energy collisions ever produced by mankind.

Even if we ignore the physics we'll observe during the collisions, it will be a huge acheivement just to get circulating proton beams at energies that high. But there are bigger, natural, accelerators in the universe, like the accretion disc of a black hole or a supernova, that can accelerate particles to thousands, millions of times the energy we're hoping for at the LHC. Some of these particles hit our atmosphere every single day, and they haven't made a black hole yet, and we don't see any evidence for this happening on any other planets either. So I'm not worried, you shouldn't be worried, really there's nothing to be worried about."

How many was that? Did he just say worried three times in one sentence? That's excessive. If you want people not to worry, you shouldn't use the word worry three times in one sentence, even if you're telling them not to do it, it's basic psychology. He considered for a moment telling them about micro black holes and that it might actually be possible to make black holes at the LHC, just not dangerous ones, but quickly decided not to. Let's have a good question.

"Why not just use those particles then?" It was that same kid. Maybe a law student.

"Which particles? The cosmic rays?" 

"The ones hitting our atmosphere." 

"Ok fair." Adam said, then directing his attention to the audience at whole. "The question is: if there are higher energy particles hitting out atmosphere for free every day, why spend millions of dollars building our own accelerator? The answer is simple: statistics. What we're looking for are rare events. We're looking for things to happen that happen far less often than once in a million. And we don't just need these events to happen somewhere randomly on the surface of the earth, we need them to happen where we can see them, where we can build entire detectors around them to capture every little detail about what took place. We want them to happen over and over again, in as nearly identical conditions as possible. Does that answer your question?"

"Yes, thank you," said the kid, and sat down. The third person didn't ask a question, but made what might loosely be called a statement. 

"XXXXXXXXXXXXXXX RAMBLING CRAZY PHYSICS THEORY XXXXXXXXXXXXXXXX" The guy sure did seem to like the word government.

\mypause

"Ok I think there's time for one more," said Sarh approaching the podium, her own microphone in hand. "Any ladies perhaps?" She could say ladies. Maybe he could too if he really owned it like she did. I don't think so. She is a lady. I'm not. A girl raised her hand and was passed a microphone. Also probably not 25 and wearing an orange sweater. People don't wear orange very much but she was fashionable too, in a Madison hipster kind of way. She didn't have gagued ears but she could have. 

"So a black hole is like a hole in space right?"

"Kind of. It's more like a place where there's so much matter in such a small area that spacetime, not just space, but the combination of space and time, becomes so curved that there is no path out that would take a finite time to travel." 

"But what about the singularity?"

"Ok yes, the thing is, nobody really knows what's happening in the middle of a black hole. By definiton, we can't see inside it, and the equations we use to describe physics start to give infinity as the answer. Some call this the singularity, some say that quantum mechanics fixes everything, some say that spacetime itself breaks down." Adam wasn't sure exactly what she was asking, and had the feeling that she wasn't completely sure either.

"Ok so my question is this: If I take a piece of aluminum and I keep on bending it, that weakens it and eventually it will break. Even if they're at higher energy, these cosmic rays are scattered all over the planet but you're making collisions in the exact same spot over and over. Could you, I don't know, like, weaken that spot?"

She was a punk too, but a good punk. "Hm, there are a few ways I could try to answer that question. I guess the lack of seeing this in other parts of the universe isn't such a convincing argument then, given our track record for finding intellegent alien life capable of producing particle accelerators."

"No," she quietly inturrupted him, speaking into her mic. He had meant that to be rhetorical.

"Then how about this. First, and it's worth not forgetting, we don't have any theoretical reasons to think that we will create a black hole that would destroy the planet. We do actually think it could be possible to create small, micro black holes, but even that is a bit of a stretch. And, thanks to Stephen Hawking, we know the mechanism by which they would evaporate in microseconds.

Second, and about weakening spacetime itself. That's an interesting thought, but consider this. The earth is rotating, it's moving around the sun, which is moving around the center of the milky way, which is accelerating toward the Andromeda galaxy and will collide in XXXXXXX billion years. So we're not exactly sitting on the same patch of land, and of course, I'm not sure if your foil analogy really holds up anyways."

\mypause

Sarah stepped beside Adam at the podium and he moved out of her way. "Well thank you so much to our Distinguished Lecturer, Professor Adam Olsen, particle physics professor down the street at UW Madison and at CERN in Geneva Switzerland, and thanks so much to all of you for coming to enjoy this wonderful space which was built by the Monona County Library Alumni Association in part using an extremely generous gift from Judith and Harold M. Butler who wanted to create a public space, for intellectuals from all walks of life to gather and discuss science and the issues of our time. And with that, let's thank our speaker once again, and enjoy some refreshments in the foyer. Thank you. Thank you." She trailed off as the audience started to clap and she joined them, turning to face Adam and then shaking his hand. 

\mymark

Not bad, all in all. 


