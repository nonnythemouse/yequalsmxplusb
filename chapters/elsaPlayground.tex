\chapter*{Elsa}
  When Elsa was a kid, she was friends with Sylvia Butts. The butt of all jokes. Both of Sylvia's father's parents were German, though she didn't look it, and had both moved to the USA separately with their families from different parts West Germany shortly after World War II. When Sylvia's great grandfather, Kristof Butz, was immigrating, the US official decided that the opportunity was too good to pass up and americanized his name and consequently Sylvia's name to Butts with the smirking approval of his supervisor. Kristof understood what was happening, but his children would grow up far away from nazis and he wasn't going to let a dispute at the boarder stop that. Kids called Sylvia names like Buttso, Silly Butt, and Sliver Butt Gorilla and she didn't hear about nazis until she was eight. 

 During recess, the jocks played kickball on the asphalt square in the zone behind the school. There was a baseball field behind that, with a fence starting at the dugout which marked in no unclear terms that this was the edge of allowed terratory. Technically the baseball diamond was a softball diamond, but the town wasn't about to build two of them and this would be less to mow and who would care anyways. Behind that was Mt. Hill, or just the Hill, technically named Bakers Hill on old contour maps that had been painstakenly and inaccurately drawn by hand back in the neolithic era before being scanned, archived, and forgotten by modern man. It was the tallest hill in town and topped with trees and lead back into a forest that went forever but had no trails. Nobody ever really explored it. Kind of out of the way, and when Elsa came back as a high schooler looking for a secret place to smoke pot with her friends, she found out that it really wasn't all that deep. Behind the woods was a housing development, then a state highway, mall, university, student housing, teacher housing which wasn't technically teacher housing but was just the neighborhood where all the young teachers lived with their young children and young spouses and nutured their young professorships, and then more developments.

  In the winter, families would come to Mt. Hill to sled, laughing, and all of the kids had been there a million times. But now they were in school, and even though it was recess there were limits, and Mt. Hill was in a different universe. It was important for children to learn that there are limits and some universes are inaccessable. In the classroom yes, but especially during recess.

Then there were those times the big kid, Zach, would toeball a home run over the fence. The teachers guarding the perimeter from their picnic table next to the door to the gym would look up from their private teacher conversations, a look of burden, of having been pulled back to earth from their lofty teacher conversations about the mysteries of the universe and the keys to the mysteries of the universe and how to best bequeath these keys to this next generation of minds before them who couldn't even play a simple game like kickball without finding a way to ensure that they had to be watched over, and therefore in a way, protected. But only in a way. A look of scolding, of boredom, of once again having to explain fractions.

  They would chase that smug kid, the one furthest back in the outfield, closest to first base on the baseball field which was really a softball field, who had been waiting for this moment for three weeks after Zach kicked his first homer and some other kid, probably that tattletale Owen, had asked one of the teachers to get it for them but wasn't fast enough this time and who took it as his duty to jump the fence and retrieve the ball, but not too quick to seem too eager and also to savor his moment on the other side like he was a celebrity strolling into a restauraunt with a dish named after him. Not an old celebrity, the kind who long ago traded their own reality which now consisted of high profile parties and interviews on camera with the reality of the screen where they played the role of normal citizens who were heros in their own way and fought the good fight and lived the lives the actors secretly longed for which incidentally made their acting all the better because their yearnings were real and felt for the first time even though they were starting to get wrinkles where they smiled, but a young actor who finished wrapping his second big role, or a rock star, or those new professors on the other side of the mall. But they would never stand up from the picnic table. They would only chase him with their eyes.

  The teachers never said anything to Kyle who jumped the fence or to Zach who kicked the ball, just like they never said anything to the boys who dropped handfuls of mulch on the heads of girls braiding each others hair on the big rickety bridge on the far side of the playground as they ran by and escaped down the slide. There was an order you didn't upset and everyone played their role even if they didn't understand it. If the boys didn't drop the mulch, what would the girls have to do with the hair? Boys and girls played tag together but there was usually also a separate boys-only-can't-touch-the-ground-tag for the semi-athletic boys who never quite got the hang of kickball but still wanted to show off. 

  On the other side of the kickball court were the swings and the nest. There was always too long a wait for the swings since nobody ever got off once they were on, and some kids would even go the whole recess without giving up their turn. Probably they were only children. But there was never a line waiting for the swings because kids are immune to futility. So while they waited they'd play tag or kickball or drop mulch in hair but never receive mulch in hair because the only times girls got the swings were when they were at the front of the line coming out of lunch because they had asked the teachers and the teachers had said ok because these girls were the teacher's pets and could do anything they wanted and never left the swings once they got on, until recess was over because they were also selfish. But when normal kids got there first, they would eventually get bored and it would take a minute before the vacancy was realized and the inevitable footrace was lead by whoever spotted it first coming over from the playground. This is why girls didn't get the swings, they weren't as fast. It was fair.

  The nest was a metal tangle of bars, curving and welded together with unquestionable fidelity, but with gaps large enough for children to fall through onto the wood chippings below. It was too far from the playground to be incorporated into anything and the kids secretly eying the swingset certainly wouldn't want to play exclusively with each other, trapped on a metal island, away from the other children, so the nest was usually abandoned until Elsa and Sylvia claimed it as their own. 

  First it was the spiderdome, and then it was the robot beehive where Robot Elsa and Robot Sylvia made robot honey out of grass and mulch to feed the robot baby bees which crawled all around the base so fast and using so much energy that they must get very hungry. Ants were the other reason most kids didn't play over there much. 

  Now it was a nest and from their perch, Elsa or Sylvia could have either easily won a swing for the small price of abandoning their friend. But they were hawks and they could fly if they wanted to. They could climb, circling the school, and then the playground. They could soar past the kickball court and the baseball diamond, not looking back to see the girls with mulch in their hair as they disappeared to a point in the background as they continued over Mt. Hill and then on to the woods that stretched to infinity. They could do this if they wanted to, but the knowledge that they could was enough so for now they were perched, watching for rabbits or field mice even though they both agreed they weren't all that hungry. %Watching and waiting like teachers waiting for another kickball over the fence.

\mymark

%\end{document}
